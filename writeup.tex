\documentclass[12pt]{amsart}

\newtheorem{theorem}{Theorem}

\usepackage[margin=1in]{geometry}

\begin{document}
    \title{Applying the Borsuk-Ulam Theorem}
    \author{Akash Dhiraj}
    %\address{}
    %\email{ad739@cornell.edu}
    \date{\today}

    \maketitle

    \newpage

    Our goal with this talk is to motivate topology more generally

    \section{Setup}

    \newpage

    \section{Borsuk-Ulam}

    \begin{theorem}
        Given a continuous map $f: \mathbb S^n \to \mathbb R^n$, $f$ identifies two antipodal points: i.e. $\exists \mathbf x \in \mathbb S^n$ such that $f(\mathbf x) = f(-\mathbf x)$.
    \end{theorem}

    \newpage

    \section{Hex Game}

    \newpage

    \section{Necklace Problem}

    \newpage


    
\end{document}