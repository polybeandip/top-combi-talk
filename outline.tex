\documentclass{beanSET}

\name{Akash Dhiraj}
\mail{ad739@cornell.edu}
\coursename{}
\courseNo{}
\semester{}
\hwNo{}
\duedate{}

\usepackage{amsthm,amsmath, amssymb}

%setup for colored theorem boxes
\usepackage[dvipsnames]{xcolor}
\usepackage{thmtools}
\usepackage[framemethod=TikZ]{mdframed}


\mdfdefinestyle{mdbluebox}{%
    linewidth=0.5pt,
    skipabove=12pt,
    frametitleaboveskip=5pt,
    frametitlebelowskip=0pt,
    skipbelow=2pt,
    frametitlefont=\bfseries,
    innertopmargin=8pt,
    innerbottommargin=8pt,
    nobreak=true,
    backgroundcolor=TealBlue!5,
    linecolor=blue,
}

\declaretheoremstyle[
    headfont=\sffamily\bfseries\color{MidnightBlue},
    mdframed={style=mdbluebox},
    headpunct={\\[3pt]},
    postheadspace={0pt}
]{thmbluebox}

\declaretheorem[style=thmbluebox,name=Theorem,numbered = no]{theorem}

\begin{document}

\begin{center}
\huge \textbf{Outline for Topoligical Combinatorics}
\end{center}

The premise of this talk would be to discuss the some of the basics of topology and then eventually state (and prove, maybe\footnote{For the sake of time, we might limit the discussion of Borsuk-Ulam to a proof sketch.}) the Borsuk-Ulam Theorem, Sperner's Lemma, and Brouwer's fixed point theorem. Following which, we present a variety of applications in combinatorics. Most of the material comes from \cite{topcombi}.\\
\\

Topological Results:

\begin{theorem}[Brower Fixed Point Theorem]
    Let $K \subseteq \mathbb R^n$ be convex and compact, and let $f: K \to K$ be continuous. Then, there exists $k \in K$ such that $f(k) = k$.
\end{theorem}

\begin{theorem}[Borsuk-Ulam Theorem]
    Suppose $f: \mathbb S^n \to \mathbb R^n$ was continuous. Then, there is some $x \in \mathbb S^n$ such that $f(x) = f(-x)$.
\end{theorem}

Combinatorial Results:

\begin{theorem}
    Every (open) necklace with $d$ kinds of jewels can be divided between two thieves using no more than $d$ cuts.
\end{theorem}

\begin{theorem}[Kneser Conjecture]
    For all $k > 0$ and $n \geq 2k - 1$, the chromatic number of the Kneser graph $KG_{n,k}$ is $\chi(KG_{n,k}) = n - 2k + 2$.
\end{theorem}

The game of Hex is played between two players, red and green. Each of them takes turns coloring one hexagon in a grid. After all hexagons are filled, red wins if there is a red path connecting the top and bottom, and green wins of there is a green path connecting the left and right.

\begin{theorem}
    The game of Hex cannot end in a draw.
\end{theorem}


\vfill

\bibliography{ref}
\bibliographystyle{plain}















\end{document}