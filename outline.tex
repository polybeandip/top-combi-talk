\documentclass{amsart}

\newtheorem{thm}{Theorem}[section]
\newtheorem{lem}[thm]{Lemma}
\newtheorem{prop}[thm]{Proposition}
\newtheorem{cor}[thm]{Corollary}

\theoremstyle{definition}
\newtheorem*{rem}{Remark}

\begin{document}

\title{Outline for Topological Combinatorics}

\maketitle

The premise of this talk would be to discuss the some of the basics of topology and then eventually state (and prove, maybe\footnote{For the sake of time, we might limit the discussion of Borsuk-Ulam to a proof sketch.}) the Borsuk-Ulam Theorem and Brouwer's fixed point theorem. Following which, we present a variety of applications in combinatorics. Most of the material comes from \cite{topcombi}.\\

\section{Setup}

\section{Results in topology}

\begin{thm}[Brower Fixed Point Theorem]
    Let $K \subseteq \mathbb R^n$ be convex and compact, and let $f: K \to K$ be continuous. Then, there exists $k \in K$ such that $f(k) = k$.
\end{thm}

\begin{thm}[Borsuk-Ulam Theorem]
    Suppose $f: \mathbb S^n \to \mathbb R^n$ was continuous. Then, there is some $x \in \mathbb S^n$ such that $f(x) = f(-x)$.
\end{thm}

\begin{thm}
    For $n \geq 0$, the following statements are equivalent:
    \begin{enumerate}
        \item For every continuous mapping $f: \mathbb S^n \to \mathbb R^n$ there exists a point $\mathbf x \in \mathbb S^n$ with $f(x) = f(-x)$.
        \item For every antipodal mapping $f: \mathbb S^n \to \mathbb R^n$, there exists a point $\mathbf x \in \mathbb S^n$ satisfying $f(\mathbf x) = 0$.
        \item There is no antipodal mapping $f: \mathbb S^n \to \mathbb S^{n-1}$.
        \item There is no continuuos mapping $f: $
    \end{enumerate}
\end{thm}   

\section{Results in Combinatorics}

\begin{thm}
    Every (open) necklace with $d$ kinds of jewels can be divided between two thieves using no more than $d$ cuts.
\end{thm}

\begin{thm}
    The game of Hex cannot end in a draw.
\end{thm}


\vfill

\bibliography{ref}
\bibliographystyle{plain}















\end{document}